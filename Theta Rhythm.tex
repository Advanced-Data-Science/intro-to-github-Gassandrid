% Options for packages loaded elsewhere
\PassOptionsToPackage{unicode}{hyperref}
\PassOptionsToPackage{hyphens}{url}
\documentclass[
]{article}
\usepackage{xcolor}
\usepackage{amsmath,amssymb}
\setcounter{secnumdepth}{-\maxdimen} % remove section numbering
\usepackage{iftex}
\ifPDFTeX
  \usepackage[T1]{fontenc}
  \usepackage[utf8]{inputenc}
  \usepackage{textcomp} % provide euro and other symbols
\else % if luatex or xetex
  \usepackage{unicode-math} % this also loads fontspec
  \defaultfontfeatures{Scale=MatchLowercase}
  \defaultfontfeatures[\rmfamily]{Ligatures=TeX,Scale=1}
\fi
\usepackage{lmodern}
\ifPDFTeX\else
  % xetex/luatex font selection
\fi
% Use upquote if available, for straight quotes in verbatim environments
\IfFileExists{upquote.sty}{\usepackage{upquote}}{}
\IfFileExists{microtype.sty}{% use microtype if available
  \usepackage[]{microtype}
  \UseMicrotypeSet[protrusion]{basicmath} % disable protrusion for tt fonts
}{}
\makeatletter
\@ifundefined{KOMAClassName}{% if non-KOMA class
  \IfFileExists{parskip.sty}{%
    \usepackage{parskip}
  }{% else
    \setlength{\parindent}{0pt}
    \setlength{\parskip}{6pt plus 2pt minus 1pt}}
}{% if KOMA class
  \KOMAoptions{parskip=half}}
\makeatother
\usepackage{graphicx}
\makeatletter
\newsavebox\pandoc@box
\newcommand*\pandocbounded[1]{% scales image to fit in text height/width
  \sbox\pandoc@box{#1}%
  \Gscale@div\@tempa{\textheight}{\dimexpr\ht\pandoc@box+\dp\pandoc@box\relax}%
  \Gscale@div\@tempb{\linewidth}{\wd\pandoc@box}%
  \ifdim\@tempb\p@<\@tempa\p@\let\@tempa\@tempb\fi% select the smaller of both
  \ifdim\@tempa\p@<\p@\scalebox{\@tempa}{\usebox\pandoc@box}%
  \else\usebox{\pandoc@box}%
  \fi%
}
% Set default figure placement to htbp
\def\fps@figure{htbp}
\makeatother
\setlength{\emergencystretch}{3em} % prevent overfull lines
\providecommand{\tightlist}{%
  \setlength{\itemsep}{0pt}\setlength{\parskip}{0pt}}
\usepackage{bookmark}
\IfFileExists{xurl.sty}{\usepackage{xurl}}{} % add URL line breaks if available
\urlstyle{same}
\hypersetup{
  pdftitle={Theta Rhythm - A memory clock},
  hidelinks,
  pdfcreator={LaTeX via pandoc}}

\title{Theta Rhythm - A memory clock}
\author{}
\date{2025-02-03}

\begin{document}
\maketitle

\emph{study notes for a \href{Artem\%20Kirsanov}{Artem Kirsanov}
\href{https://youtu.be/5CxSoFK5tOQ?si=xNx2_g403li49K1i}{video}}.

Recently, there have been some significant advancements in decoding the
hippocampus, and how it is involved in memory. This is a very complex
structure, and it is not yet fully understood. However, we do know that
the hippocampus is involved in the formation of new memories, and the
retrieval of old ones.

More specifically, looking at the brain's internal clock, the
\textbf{Theta Rhythm} is a very interesting phenomenon. This is a brain
wave that is present in the hippocampus, and it is associated with
\href{Memory\%20Formation}{Memory Formation} and retrieval. We believe
that the theta rhythm is a clock that helps the brain to organize
memories in time.

\begin{center}\rule{0.5\linewidth}{0.5pt}\end{center}

\subsection{Brain Waves}\label{brain-waves}

Let's start by coming to understand what the brains internal waves
actually are.

\textbf{Brain Waves} were first discovered in the 1920s by a German
psychiatrist named Hans Berger. He discovered that the brain emits
electrical signals, and that these signals can be measured using an
\textbf{Electroencephalogram (EEG)}, essentially some electrodes
attached to someone's head. This is a device that measures the
electrical activity of the brain.

What was shown was several different oscillating patterns, which we now
call \textbf{Brain Waves}. These waves are classified by their
frequency, and they are associated with different states of
consciousness.

At this point in history, we know quite a lot about the different kinds
of brain waves, and the frequencies that they operate at. For example,
\textbf{Delta Waves} are associated with deep sleep, \textbf{Alpha
Waves} are associated with relaxation, and \textbf{Beta Waves} are
associated with alertness.

\begin{figure}
\centering
\pandocbounded{\includegraphics[keepaspectratio,alt={Pasted image 20250324102549.png}]{/Users/gassandrid/Desktop/Pasted image 20250324102549.png}}
\caption{Pasted image 20250324102549.png}
\end{figure}

But for this note, we are interested in the \textbf{Theta Rhythm}. This
is a wave that operates at a frequency of around 4-8 Hz, and it is
associated with the hippocampus.

\begin{center}\rule{0.5\linewidth}{0.5pt}\end{center}

\subsection{Generation of Theta
Rhythm}\label{generation-of-theta-rhythm}

The theta rhythm is characteristically described by its wavelength, with
a frequency between \textbf{4} and \textbf{12} Hz.

It is measured by sticking an electrode inside the hippocampus and
measure the voltage difference between a point in
\href{Extracellular\%20Space}{Extracellular Space} and a ground
electrode ( typically in the base of the skull ).

This voltage arises out of the summation of currents of neurons,
reflecting synchronous synaptic inputs.

That is, when \textbf{one neuron} sends information to another through
synapses.

\begin{figure}
\centering
\pandocbounded{\includegraphics[keepaspectratio,alt={Screenshot 2025-02-03 at 1.53.13 PM.png}]{/Users/gassandrid/Desktop/Screenshot 2025-02-03 at 1.53.13 PM.png}}
\caption{Screenshot 2025-02-03 at 1.53.13 PM.png}
\end{figure}

In the case where there is a large number of neurons surrounding this
electrode, and they are constantly sending signals simultaneously, then
these individual currents \textbf{add up}, leading to the voltage
becoming strong enough to be sensed by external electrode.

\begin{quote}
{[}!Warning{]} Note Remember this measurement is finding the
\textbf{overall activity} sensed by the electrode, and not the behavior
of any individual neuron. As \href{Artem\%20Kirsanov}{Artem} put it,
this is like putting a microphone over a stadium, you cant discern
individual phenomena but rather general group behavior patterns(which
are still useful).
\end{quote}

This special rhythm is generated in a special structure, called the
\href{Medial\%20Septum}{Medial Septum}. This structure contains a large
amount of something called \href{Pacemaker\%20Neurons}{Pacemaker
Neurons}, which are known to regularly discharge something between 4 and
12 times a second.

These special neurons have some special proteins called
\href{Hyperpolarization\%20Activated\%20Channels}{Hyperpolarization
Activated Channels}(HCNs), which allow Ions to flow into the cell at a
regulated time interval, generating this rhythmic pattern.

\begin{figure}
\centering
\pandocbounded{\includegraphics[keepaspectratio,alt={Screenshot 2025-03-26 at 12.07.01 PM.png}]{/Users/gassandrid/Desktop/Screenshot 2025-03-26 at 12.07.01 PM.png}}
\caption{Screenshot 2025-03-26 at 12.07.01 PM.png}
\end{figure}

\begin{quote}
{[}!Important{]} Fun Fact These are the same proteins that your heart
uses to regulate its beating rhythm.
\end{quote}

These \href{Pacemaker\%20Neurons}{Pacemaker Neurons} then send their
connections straight to the \url{Hippocampus}, thus providing the
rhythmic output and leading to the rise of the theta wave. This causes
the activity of the Hippocampus to wax and wane, following the rhythm
that the conductor(\href{Medial\%20Septum}{Medial Septum}) is providing.

\begin{center}\rule{0.5\linewidth}{0.5pt}\end{center}

\begin{quote}
{[}!Warning{]} Quick Note again\ldots{} This is quite the
oversimplification of an otherwise very complex process. As
\href{Artem\%20Kirsanov}{Artem} mentions, a group of researchers led by
Romain Goutagny\footnote{Goutagny R, Jackson J, Williams S.
  Self-generated theta oscillations in the hippocampus. Nat Neurosci.
  2009 https://pubmed.ncbi.nlm.nih.gov/19881503/} found that under
certain conditions, the Hippocampus itself can generate theta rhythm,
indicating that it contains some neuronal machinery sufficient enough
for the generation of intrinsic rhythm.
\end{quote}

In this scenario, it would mean that the hippocampus, much like a well
trained orchestra, can perform coherently even in the absence of its
conductor.

Goutagny Et al.\footnote{Goutagny R, Jackson J, Williams S.
  Self-generated theta oscillations in the hippocampus. Nat Neurosci.
  2009 https://pubmed.ncbi.nlm.nih.gov/19881503/} found that rhythmic
activity arises due to the interactions between excitatory and
inhibitory neurons, through a negative feedback loop.

\begin{figure}
\centering
\pandocbounded{\includegraphics[keepaspectratio,alt={Screenshot 2025-03-26 at 12.18.34 PM.png}]{/Users/gassandrid/Desktop/Screenshot 2025-03-26 at 12.18.34 PM.png}}
\caption{Screenshot 2025-03-26 at 12.18.34 PM.png}
\end{figure}

\footnote{Hummos A, Nair SS. An integrative model of the intrinsic
  hippocampal theta rhythm. Lytton WW, editor. PLoS ONE. 2017
  https://pubmed.ncbi.nlm.nih.gov/28787026/}

It is quite an elegant mechanism, however a tad bit complex for an
introduction to Theta Rhythm. I will most likely create a note later on
taking a
\href{Self-generated\%20theta\%20oscillations\%20in\%20the\%20hippocampus}{deeper
look at the paper alone}.

Aside from that quick detour, let us assume that for the most part, the
\href{Medial\%20Septum}{Medial Septum} is responsible for the beat
generation for the \url{Hippocampus}. Even though the hippocampus does
have an \textbf{intrinsic oscillator} within, a bulk of the computation
will come from the \textbf{upstream oscillation} of the medial septum.

\begin{figure}
\centering
\pandocbounded{\includegraphics[keepaspectratio,alt={Screenshot 2025-03-26 at 12.29.47 PM.png}]{/Users/gassandrid/Desktop/Screenshot 2025-03-26 at 12.29.47 PM.png}}
\caption{Screenshot 2025-03-26 at 12.29.47 PM.png}
\end{figure}

\begin{center}\rule{0.5\linewidth}{0.5pt}\end{center}

\subsection{Functions of the Theta
Wave}\label{functions-of-the-theta-wave}

Just like any other brain rhythm, Theta is not always present.

Theta Rhythm is observed under some specific physiological states,
namely periods of locomotion(aka running), exploratory sniffing, and
various environment related behaviors.o

One example might be when an animal is grooming itself, it is unlikely
that Theta activity would be present at all, or at the very least, under
very sparse bursts. You wouldn't note the slow frequency component in
this case.

But when this animal starts engaging in an activity such as running or
for the search of food, it's \url{Hippocampus} would light up the
rhytmic theta wave.

\href{Spacial\%20Navigation}{Spacial Navigation} is closely linked to
\href{Memory\%20Formation}{Memory Formation}(as they are both functions
of the Hippocampus), and this makes sense for a lot of creatures. Say
you are a mouse, and you find a food source by taking a specific route
to find it. It is evolutionarily beneficial to take special note of a
route like this so that the mouse can retrieve it next time.

\begin{quote}
While the specifics are up for discussion,
\href{Artem\%20Kirsanov}{Artem} seeks to make the argument that the
central purpose of \textbf{theta rhythm} is to provide a computational
mechanism to \textbf{construct} and \textbf{retrieve} memory traces.
\end{quote}

For the purpose of explaining what he means here, let's think of a
\url{Memory} as a \textbf{temporal} sequence of neuronal patterns
activating one after another. Each one of these patterns is a
\textbf{collection} of neuronal assemblies, which together form an
\textbf{integrated representation} of the world.

\begin{figure}
\centering
\pandocbounded{\includegraphics[keepaspectratio,alt={Screenshot 2025-04-05 at 2.43.48 PM.png}]{/Users/gassandrid/Desktop/Screenshot 2025-04-05 at 2.43.48 PM.png}}
\caption{Screenshot 2025-04-05 at 2.43.48 PM.png}
\end{figure}

This ``integrated representation'' has the feature of encoding different
modalities of a given experience.

This could be things like the \textbf{current position}, \textbf{sensory
cues}, the \textbf{emotional state} at that time, the \textbf{social
interactions} in that given place/experience\ldots{} The list goes on.

Each of these ``modalities'' has a \textbf{corresponding pattern} of
brain activity, a ``neural fingerprint'' of sorts that uniquely
identifies this given information source.

A great example of this is another video made by Artem(which I plan on
writing a note on, if not done already) about
\href{Place\%20cells:\%20How\%20your\%20brain\%20creates\%20maps\%20of\%20abstract\%20spaces}{Place
Cells}, which are neurons that essentially associate certain behavior
with a given location. Each of these Place Cells has a \textbf{preferred
location} in which they tend to fire. Bringing all these place cells
together, we get something of a ``code'' of physical position.

\begin{center}\rule{0.5\linewidth}{0.5pt}\end{center}

However, the \url{Hippocampus} has been shown to encode much more than
just spatial variables, things like \textbf{sound frequencies}, and even
the \textbf{identities} of certain people(in the case of humans at the
least).

These all come together to form a modality rich experience that
represents all of the different sources of information for a given
episodic memory.

But to form an integrated representation for the external world, the
brain has the problem of \textbf{unifying} these different modalities
into a single ``picture''. \textbf{Theta Rhythm} has the purpose of
ensuring that all these separate components link together to form a
unified memory.

In addition to the ``assembly'' of a unified memory, another problem
that the brain has to solve is the arrangement of these instantaneous
representations into a meaningful \textbf{temporal sequence}, to form a
meaningful memory which unfolds in time.

\begin{figure}
\centering
\pandocbounded{\includegraphics[keepaspectratio,alt={Screenshot 2025-04-05 at 3.19.03 PM.png}]{/Users/gassandrid/Desktop/Screenshot 2025-04-05 at 3.19.03 PM.png}}
\caption{Screenshot 2025-04-05 at 3.19.03 PM.png}
\end{figure}

This so-called \textbf{sequence organization}(a one dimensional process)
is vital for \href{Spacial\%20Navigation}{Spacial Navigation}, as each
trajectory/path is made up of a sequence of activated place cells. But
it is also vital for \textbf{episodic memory} in general. Theta Rhythm
helps in the process of forming these ``chains of representation'',
ensuring the correct temporal order.

\begin{center}\rule{0.5\linewidth}{0.5pt}\end{center}

\subsection{Forming an Integrated
Representation}\label{forming-an-integrated-representation}

To cement these concepts properly, lets work with an example:

\begin{quote}
{[}!Example{]} \textgreater{} \textbf{Alice} and \textbf{Bob} are
childhood friends who have not seen each other in a very long time.

One day, they randomly bump into each other in the bank. But, given that
both were busy, they thought it best to go and meet up later at their
\textbf{favorite coffee shop} to catch up with each other.

Suppose they dont have the luxuries we do now ( the internet, phones,
etc ) to communicate with each other, other than face to face
interaction. How can they possibly make the meeting at the coffee shop
happen?
\end{quote}

\^{}85a4a4

One way is to agree on the place to meet like they did, and then
occasionally visit it once in a while. There is a small chance that they
meet each other there by coincidence, where they can finally sit down
and talk. But we can all agree that this is a sub-par method of ensuring
contact.

A more intuitive way to us is to \textbf{agree on a time} in advance. If
they both know that they are meeting each other at \emph{1pm} tomorrow
at their favorite coffee shop, there is no ambiguity left. In layman
terms, even if they don't communicate directly, they can use an
\textbf{external factor}, in the form of time, which they both have
access to in order to \textbf{coordinate} their arrival.

\begin{figure}
\centering
\pandocbounded{\includegraphics[keepaspectratio,alt={Screenshot 2025-04-06 at 4.16.18 PM.png}]{/Users/gassandrid/Desktop/Screenshot 2025-04-06 at 4.16.18 PM.png}}
\caption{Screenshot 2025-04-06 at 4.16.18 PM.png}
\end{figure}

\begin{quote}
{[}!Abstract{]} Note In the ancient times before clocks were around,
people used the position of the sun for the exact same purpose
\end{quote}

In the \url{Hippocampus}, the Theta Rhythm, and in particular it's
\textbf{phase}(between \(0^\circ\) and \(360^\circ\)) is used as this so
called ``clock''. An \textbf{external agent''}, which all \url{Neuron}s
can listen to, and coordinate their spiking activity.

\begin{figure}
\centering
\pandocbounded{\includegraphics[keepaspectratio,alt={Screenshot 2025-04-06 at 4.50.54 PM.png}]{/Users/gassandrid/Desktop/Screenshot 2025-04-06 at 4.50.54 PM.png}}
\caption{Screenshot 2025-04-06 at 4.50.54 PM.png}
\end{figure}

Because, for the brain to ``link'' assemblies together(such as ones
representing \textbf{where} and \textbf{with whom}), spikes from their
neurons should arrive at the receiver neuron in a very small time
window, with a very short interval between each other. That way, theta
oscillation provides a \textbf{temporal reference} which can be used by
neurons to adjust the timings of their spikes in order to caryr out
information transfer more effectively.

The reason why theta rhythm activates during exploratory
behavior(running, etc), is that spacial behavior needs navigation and
memory encoding. For this reason, our brain requires that we constantly
form \textbf{integrated representations} of environments.

Going back to \href{Theta\%20Rhythm\#\%5E85a4a4}{our example}, this
means that all of a sudden, everyone has a lot of catching up to do with
their old friends. Therefore, the \textbf{global timing mechanism} is
used for hundreds of similar people to coordinate their arrivals
effectively en masse.

\emph{However, this only really represents about half of the purpose of
theta rhythm, as we will learn}

At this point, these constructed amalgamations of information sources is
only a stationary snapshot of the external world at some point in time.
It does not encode the past or future temporal order into the episodic
memory. It is the \textbf{chaining} of these neuronal assemblies into a
temporal sequence that of better use to the brain.

\emph{this is like taking a freeze frame of a movie!}

\begin{center}\rule{0.5\linewidth}{0.5pt}\end{center}

\subsection{Sequential Organization}\label{sequential-organization}

To achieve this \textbf{chaining of assemblies}, there are two possible
mechanisms we can use.

\begin{enumerate}
\def\labelenumi{\arabic{enumi}.}
\tightlist
\item
  The first one are \textbf{externally generated} sequences. they arrise
  when the incoming information \emph{already} has a sequencial
  organization.
\end{enumerate}

\begin{quote}
{[}!Example{]} suppose you(a mouse) are walking through a corridor with
a rainbow colored gradient wall. In the beggining its red, middle its
yetllow, end is blue, etc.

In that case, visual information, which is organized as a succesion of
changing colors, evokes a chain corresponding representations in the
brain

\begin{figure}
\centering
\pandocbounded{\includegraphics[keepaspectratio,alt={Screenshot 2025-04-06 at 7.09.12 PM.png}]{/Users/gassandrid/Desktop/Screenshot 2025-04-06 at 7.09.12 PM.png}}
\caption{Screenshot 2025-04-06 at 7.09.12 PM.png}
\end{figure}
\end{quote}

\begin{enumerate}
\def\labelenumi{\arabic{enumi}.}
\setcounter{enumi}{1}
\tightlist
\item
  The alternative and much more interesting option is that of the
  \textbf{internally generated} sequences. This would likely emerge when
  there are little to no incoming sensory information, and thus it is
  left up to the brain to organize sequentially
\end{enumerate}

\begin{quote}
{[}!Example{]} Imagine if that same mouse has sat to lie down with its
eyes closed, and is trying to recall the experience of walking through
that rainbow gradient corridor. In this case, the sequence
representation has to be generated by the brain itself. Connectivity
patterns and synaptic strengths between members of the assembly itself.
Aka, if the activation of the first assembly leads to the second, and so
on and so fourth.

\begin{figure}
\centering
\pandocbounded{\includegraphics[keepaspectratio,alt={Screenshot 2025-04-06 at 7.20.02 PM.png}]{/Users/gassandrid/Desktop/Screenshot 2025-04-06 at 7.20.02 PM.png}}
\caption{Screenshot 2025-04-06 at 7.20.02 PM.png}
\end{figure}
\end{quote}

\^{}4f4e3b

The interesting thing here is that it turns out that \textbf{Theta
Rhythm} itself is \emph{essential} for the generation
\href{Theta\%20Rhythm\#\%5E4f4e3b}{the latter type}.

When the Theta Rhythm is abolished in test subjects, it severely affects
the emergence of this type of sequential activity, which is normally
observed during planning and memory retrieval.

Let's put this in better terms by returning to the
\href{Theta\%20Rhythm\#\%5E85a4a4}{cafe example}:

\begin{quote}
{[}!Example{]} Now lets suppose that all these ``pair encounters''
between many different \textbf{Alices} and \textbf{Bobs} should happen
in a \textbf{strict temporal succession}. For this case, lets suppose
this takes the form of a \emph{couples} therapy.

First, Alice and Bob go at \texttt{10AM}. Then, Claire and Dan go after
them at \texttt{11AM}, and so fourth.
\end{quote}

In this case, just like we saw before, \textbf{time serves to coordinate
arrivals of individuals} within a couple. But, at the same time,
\textbf{external clock} is used to separate different sessions in time,
to ensure they happen one after another in a well defined order.

\begin{figure}
\centering
\pandocbounded{\includegraphics[keepaspectratio,alt={Screenshot 2025-04-07 at 9.46.48 AM.png}]{/Users/gassandrid/Desktop/Screenshot 2025-04-07 at 9.46.48 AM.png}}
\caption{Screenshot 2025-04-07 at 9.46.48 AM.png}
\end{figure}

Otherwise, it would be somewhat awkward if \emph{all} the couples showed
up for their session at the same time.

In the same way, \textbf{Theta Rhythm} provides a mechanism to arrange
experience into a \textbf{temporal sequence}. And just like in the
previous case, it is the \textbf{phase} of the wave which serves as the
\textbf{coordination parameter}.

Because neurons adjust timings of their spikes to the ticking of the
\href{Hippocampus}{hippocampal} clock, episodic memories can unfold in a
\emph{meaningful} temporal succession, instead of being jumbled
together.

\begin{center}\rule{0.5\linewidth}{0.5pt}\end{center}

\subsection{Phase Precession}\label{phase-precession}

We have already talked about how a \textbf{path} through an environment
can be represented sequence of
\href{Place\%20cells:\%20How\%20your\%20brain\%20creates\%20maps\%20of\%20abstract\%20spaces}{place
cell} activations(the specialized neurons that encode physical location
and subsequent behavior).

Each of those cells has a preffered patch of ground, called the
\textbf{place field}, where if fires most actively.

When the animal enters the place field, reaches center, and leaves it,
the \textbf{firing rate}(see
\href{1.2\%20-\%20Spike\%20Trains\%20and\%20Firing\%20Rates}{textbook})
of that neuron would gradually increase. In reaches its maximum value at
the \textbf{center} of the place field, and then gradually decreases.

\begin{figure}
\centering
\pandocbounded{\includegraphics[keepaspectratio,alt={Screenshot 2025-04-09 at 4.26.15 PM.png}]{/Users/gassandrid/Desktop/Screenshot 2025-04-09 at 4.26.15 PM.png}}
\caption{Screenshot 2025-04-09 at 4.26.15 PM.png}
\end{figure}

This phenomenon is what we like to call \textbf{Rate Coding}, where the
information about the position is directly encoded in whatever neuron is
currently spiking, and the \textbf{firing rate} of said neuron.

\begin{figure}
\centering
\pandocbounded{\includegraphics[keepaspectratio,alt={Screenshot 2025-04-09 at 4.28.20 PM.png}]{/Users/gassandrid/Desktop/Screenshot 2025-04-09 at 4.28.20 PM.png}}
\caption{Screenshot 2025-04-09 at 4.28.20 PM.png}
\end{figure}

However, it turns out that the phase of \textbf{Theta Rhythm} at which
these neurons spike also plays a role in the information transfer. This
is known as \textbf{phase coding}.

More specifically, if one was to record the activity of individual
neurons in addition to the surrounding Theta Rhythm, you would notice
some very bizarre behavior. As the neuron moves through the place field
of a neuron, its spikes would occur at early and earlier phases of
Theta.

\begin{figure}
\centering
\pandocbounded{\includegraphics[keepaspectratio,alt={Screenshot 2025-04-09 at 4.31.48 PM.png}]{/Users/gassandrid/Desktop/Screenshot 2025-04-09 at 4.31.48 PM.png}}
\caption{Screenshot 2025-04-09 at 4.31.48 PM.png}
\end{figure}

It gets even more interesting when you consider this \textbf{Phase.
Precession} when accounting for multiple neurons. If you look at how
multiple place cells \textbf{spike} relative to each other, you will see
that the \textbf{order of spikes} within one \textbf{Theta Cycle}(The
time window between two peaks) \textbf{Recapitulates} the exact order of
place fields on the portion of the trajectory.

\begin{figure}
\centering
\pandocbounded{\includegraphics[keepaspectratio,alt={Screenshot 2025-04-09 at 4.36.17 PM.png}]{/Users/gassandrid/Desktop/Screenshot 2025-04-09 at 4.36.17 PM.png}}
\caption{Screenshot 2025-04-09 at 4.36.17 PM.png}
\end{figure}

Looking at this visualization, you will notice that there the most
overlap at the exact center of the ``red'' spike firing rate. This peak
firing rate is observed at the \textbf{trough}(valley) of the Theta
wave, at around the \(180^{\circ}\) mark.

The ``recent past'' place cells of yellow is gradually shifted to the
left as time passes, and you will notice at the end it is roughly at the
\(0^{\circ}\) mark of the Theta Wave.

Subsequently, the place cells that are just at the beginning of their
firing arc represent the \textbf{near future} neuronal spikes at the
\(360^{\circ}\) mark.

\begin{figure}
\centering
\pandocbounded{\includegraphics[keepaspectratio,alt={Screenshot 2025-04-09 at 4.58.28 PM.png}]{/Users/gassandrid/Desktop/Screenshot 2025-04-09 at 4.58.28 PM.png}}
\caption{Screenshot 2025-04-09 at 4.58.28 PM.png}
\end{figure}

This way, every cycle of \textbf{Theta} bears some sort of information
about the past, present, and future, in proper order. As the animal
moves and Theta Rhythm progresses, the representations would shift
accordingly along the wave to maintain this ``time-phase'' relationship.

It is a very interesting little system of temporal organization which
sorts segments of experiences.

It is believed that these are pieces of a much bigger puzzle, which are
referred to as ``Theta Sequences''. These sequences are likely stitched
together to encode for a full memory system.

\begin{center}\rule{0.5\linewidth}{0.5pt}\end{center}

\subsection{Conclusion/Summary}\label{conclusionsummary}

To summarize, the \textbf{Theta Rhythm} is a very interesting phenomenon
that is observed in the \url{Hippocampus}. It is a brain wave that
operates at a frequency of around 4-8 Hz, and it is associated with
memory formation and retrieval.

Integrated Representation of the external world is achieved by the
\textbf{Theta Rhythm}, which serves as a \textbf{global timing
mechanism} for the \url{Hippocampus}. It helps to link different
modalities of information together, and to organize them into a
meaningful temporal sequence.

And, to top it all off, this \textbf{Theta Wave} is also responsible for
the \textbf{Phase Precession} of *\emph{Place Cells}, which allows the
brain to encode information about the past, present, and future in a
meaningful way.

\begin{center}\rule{0.5\linewidth}{0.5pt}\end{center}

\end{document}
